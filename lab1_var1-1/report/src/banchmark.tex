\section{Тест производительности}
Для оценки времени работы были сгенерированы входные данные разных размеров. 
Так как диапазон ключей фиксирован, сортировка подсчётом показывает линейный рост времени 
по числу элементов и не зависит от распределения ключей.

Теоретически для данного варианта $k = 65536$, поэтому:
\[
T(n) = O(n + 65536) = O(n).
\]

На практике это означает, что при росте входного массива в 10 раз время работы 
увеличивается примерно в 10 раз (с учётом постоянной составляющей и накладных расходов системы).

Память под массив счётчиков фиксирована и составляет $65536 \cdot sizeof(int)$ байт,
дополнительно используется выходной массив размера $n$.

\pagebreak
