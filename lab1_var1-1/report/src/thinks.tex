\section{Выводы}
В этой лабораторной работе я реализовал стабильную сортировку подсчётом для пар
\enquote{ключ-значение} и на практике разобрал полный цикл алгоритма: подсчёт частот,
построение префиксных сумм и распределение элементов по итоговым позициям.

Главный технический вывод: линейная оценка $O(n + k)$ достигается только при заранее известном и
ограниченном диапазоне ключей. В нашем варианте это выполнимо ($k = 65536$), поэтому алгоритм
работает предсказуемо по времени даже на больших входах.

Также стало понятно, как обеспечивается стабильность сортировки: исходный массив нужно обходить
справа налево, иначе относительный порядок элементов с одинаковыми ключами нарушается.
Отдельно были отработаны требования к формату ввода-вывода для автоматической проверки.

Таким образом, сортировка подсчётом является хорошим практическим выбором для задач с
целочисленным ключом из небольшого диапазона; при слишком большом диапазоне ключей затраты памяти
становятся её основным ограничением.
\pagebreak
