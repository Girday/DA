\section{Выводы}
В лабораторной работе была реализована стабильная сортировка подсчётом для пар
\enquote{ключ-значение} и проведено сравнение со \texttt{std::stable\_sort} на серии входов разного размера.

Практический результат оказался ожидаемым: на очень малом входе сортировка подсчётом проигрывает
из-за постоянной стоимости обработки диапазона ключей ($k=65536$), но начиная с $10^4$ элементов
стабильно выигрывает по времени. На больших объёмах данных преимущество становится заметным и в
эксперименте достигает нескольких раз.

Стабильность в counting sort обеспечивается обратным проходом по исходному массиву;
линейная сложность $O(n + k)$ даёт выигрыш только при ограниченном диапазоне ключей;
сравнение производительности корректно делать по серии прогонов и медиане, поскольку абсолютные
времена чувствительны к платформе, нагрузке и деталям реализации бенчмарка.

Итог: сортировка подсчётом --- эффективный инструмент для задач с целочисленным ключом из
небольшого фиксированного диапазона, если допустимы дополнительные затраты памяти под счётчики и
временный массив результата.
\pagebreak
